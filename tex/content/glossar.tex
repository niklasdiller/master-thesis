Nachfolgend sind noch einmal wesentliche Begriffe dieser Arbeit
zusammengefasst und erläutert. Eine ausführliche Erklärung findet
sich jeweils in den einführenden Abschnitten sowie der jeweils
darin angegebenen Literatur. Das im Folgenden im Rahmen der
Erläuterung verwendete Symbol \this bezieht sich jeweils auf den
im Einzelnen vorgestellten Begriff, das Symbol \siehe{} verweist
auf einen ebenfalls innerhalb dieses Glossars erklärten Begriff.

\begin{description}
\item[Auktion] Eine \this ist das im \siehe{E-Commerce} am
Häufigsten eingesetzte Verfahren zur dynamischen
\siehe{Preisfindung}. Interessenten können dabei durch Abgabe von
Geboten Preis, Dauer und Gewinner beeinflussen. Bei einer offenen
\this sind Bieter, Höhe der Gebote und der aktuelle Preis für
alle Teilnehmer sichtbar, bei der geschlossenen (sealed) \this
erfolgt nur eine interne Benachrichtigung. Die bekanntesten Typen
sind die traditionelle \siehe{Versteigerung} sowie die
\siehe{holländische}, \siehe{umgekehrte} und \siehe{verdeckte} \this.

\item[Behaviorismus] Der \this ist eine \siehe{Lerntheorie}, die
davon ausgeht, dass Wissen als Struktur unabhängig vom
\siehe{Lernenden} existiert und dass sein Verhalten operant
konditioniert ist, d.h. dass es als Konsequenz aus anderen
Verhaltensweisen resultiert. Erfolgt eine positive Reaktion,
behält der \siehe{Lernende} neu erlerntes Verhalten bei, negative
Reaktionen führen zu einer Verminderung dieses Verhaltens. Der
\siehe{Lehrende} bestimmt dabei das zu erlernende Wissen und
ist für die Steuerung des \siehe{Lernprozesses} zuständig.
\end{description}
