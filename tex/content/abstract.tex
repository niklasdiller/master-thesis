\chapter*{Abstract}
 
 
 Accomplishing an appropriate trade-off between a machine learning model's performance and resource awareness is a challenge that increasingly necessitates the attention of data scientists and engineers, especially in edge computing. Ensuring that a certain deployed model efficiently utilizes its resources is crucial for minimizing battery consumption, managing computational resources, and freeing up network bandwidth. Particularly when working with large model databases, it can prove to be rather cumbersome to select models that fulfill this requirement in addition to providing accurate predictions. This task only gets more complex when working with sets of different models, as the number of possible model combinations exponentially increases the more learners are considered for set creation.
  
 Largely motivated by current research projects in cattle activity recognition and parking occupancy prediction, this work explores approaches to balance performance and resource awareness in the model selection process and introduces a model set retrieval system that allows for detailed customizable settings in a multi-object optimization context. After presenting essential theoretical concepts about the top-k algorithms Fagin's Algorithm (FA) and Threshold Algorithm (TA), a paradigm of different scores is constructed that aims to depict the performance and resource awareness of a model set concurrently and precisely. 
 
  For the retrieval system to work on actual objects, a large number of models have been trained and stored by using a preprocessing and training pipeline that was developed for this work. The user is then given the option to choose one of three top-k algorithms amongst several other settings to have a model set based on their preferences returned to them. To assess resource awareness in the model sets, the concept is split up into intra- and inter-model resource awareness. While the extent of efficient resource utilization among multiple models is measured by ascertaining commonalities between them using Query Sharing Levels (QSLs), single-model metrics like the number of features a model was trained on or its window size are used as a first indicator for a model's resource awareness. This distinction between intra- and inter-model resource awareness unlocks extensive potential for the model selection process, making it more precise and customizable for the user. Subsequently, a large number of models can be filtered out during the selection process, before model set building has even begun.
 
 The implemented retrieval system is finally evaluated on the metrics execution time, number of accesses, and network utilization. It is found, that the two established top-k algorithms, FA and TA, both outperform a naïve approach of selecting the best model sets, regarding execution time. TA in addition required the least amount of accesses out of all three investigated algorithms. Examining the network utilization of different model sets revealed that the fact that multiple models share certain attributes does not automatically imply a resource-aware model set. Measuring intra-model resource awareness proves to be an essential part of assessing the degree of efficient resource utilization in model sets.